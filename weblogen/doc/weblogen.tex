\documentclass{report}
\title{Weblogen Documentation}
\author{Dan Elphick}
\date{\today}

\begin{document}
\maketitle
\pagenumbering{roman}
\setcounter{page}{2}
\tableofcontents

\chapter{Introduction}
\pagenumbering{arabic}

Weblogen is a web interface to Logen. Logen is a partial evaluator for the
Prolog programming language. The purpose of the system is to illustrate the
power of partial evaluation without requiring users to download and install
Logen.

Logen can be used in several ways:

\begin{itemize}
\item Directly in Prolog.
\item Using the Python/Tk interface pylogen.
\item Using the web interface.
\end{itemize}

The first approach is the most powerful since all annotation features are
available by directly editing annotation files. The drawback is that
annotation is slow and cumbersome and requires the user to fully understand
the syntax of Prolog in order to make changes. Using the BTA mitigates this
problem, but this is only effective up to a point.

Using pylogen allows a user to perform partial evaluation without ever
looking at an annotation file. Instead annotations are manipulated using
menus triggered by clicking on predicates in a program. As long as pylogen is
updated to utilise all of Logen then it is by far the most convenient
interface for users with local access to Sicstus Prolog. However since
Sicstus is not free and Logen has not (yet) been ported to Ciao, a web
interface can make Logen more widely available.

\chapter{Usage}

The opening page gives the user two options:
\begin{itemize}
\item Upload and specialise their own file.
\item Specialise a stock file provided on the server.
\end{itemize}

Once a source file has been specified, annotations must be supplied. Weblogen
provides several ways to do this:

\begin{itemize}
\item Upload an annotation file to the server.
\item Apply a Binding-Time Analysis (BTA) to the source file.
\item Use a stock annotation file. (Only available if the source file was
provided by the server).
\end{itemize}

The annotated program will then be displayed with each of the annotations
indicated by colouring predicates appropriately.

The program can now be partially evaluated by giving a specialisation goal.
A generating extension is then produced and run using the specialisation
goal. The resulting program is then shown to user with the option to save it.

The annotations in a program can be modified by clicking on them. This will
toggle them between \emph{call} and \emph{rescall} for built-in predicates
and \emph{unfold} and \emph{memo} for others. The result can be saved by
clicking Upload annotations. Clicking ``Edit the filters'' loads a new page
where the filters are displayed in an editable box.

\chapter{Design}

\section{Sessions}

Weblogen is necessarily stateful and so must use cookies to identify
connections. Currently all state is preserved using the session facilities of
PHP. This may prove not to be scalable for large files and/or with large
numbers of users, but has proven workable.

The session stores the following information (excluding information required
to check for session existence):
\begin{itemize}
\item plfile - The current prolog program.
\item filename - The current program file.
\item annfile - The annotation file for the current program.
\item goal - The current specialisation goal.
\item specfile - The result of specialising the program.
\item filter - The initial filter used by the BTA.
\item norm - The norm used by the BTA.
\item filter\_prop - Flag indicating whether to just propagate filters or to
do a full BTA.
\item filters - The filter part of the annotation file.
\item stock - Flag indicating whether the file was a stock file.
\item stock\_annfile - Flag indicating whether the annotation file was a stock
file.
\end{itemize}

The session is automatically created once a stock prolog file is chosen or a
file is uploaded. This file is stored in \emph{plfile} and \emph{stock} is
set appropriately. 


\end{document}
